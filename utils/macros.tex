% renew commands

\renewcommand\thepage{\romannumeral\numexpr\value{page}-1\relax}

\renewcommand{\glossarypreamble}{\vspace*{\baselineskip}\vspace*{\parskip}}

\renewcommand{\arraystretch}{1.2}

% glossary custom style: 
% https://tex.stackexchange.com/questions/327328/glossaries-style-change-add-units
\newglossarystyle{betterglossary}{%
	\setglossarystyle{list}%

	\renewcommand*{\glossentry}[2]{%

		\item[\glsentryitem{##1}\glstarget{##1}{\glossentryname{##1}}]

		\glossentrydesc{##1} ##2
	}%

	\renewcommand*{\glsgroupskip}{}%
}

%\renewcommand because amssymb package has a command called \bold
\renewcommand{\bold}[1]{\textbf{#1}}

% better parallel symbol
\renewcommand{\parallel}{\mathbin{\!/\mkern-5mu/\!}}

% system modifications
 
\setlength{\heavyrulewidth}{0.07em}
\newcommand{\otoprule}{\midrule[\heavyrulewidth]}

\newdateformat{monthyeardate}{\monthname[\THEMONTH] \THEYEAR}
\newcommand{\monthtoday}{\date{\monthyeardate\today}}

% https://tex.stackexchange.com/questions/112932/today-month-as-text
\newdateformat{coverdate}{\monthname[\THEMONTH] \THEYEAR}

% new commands

\newcommand{\ket}[1]{$\left|{#1}\right\rangle$} 

\newcommand{\bra}[1]{$\left\langle{#1}\right|$}

%\newcommand{\braket}[2]{$\left\langle{#1}\middle|{#2}\right\rangle$}

\newcommand{\fract}[2]{$\frac{#1}{#2}$}

\newcommand{\matrixbraket}[2]{
	$\begin{pmatrix}
	#1 \\
	#2 \\
	\end{pmatrix}$
}

\newminted{mumax3}{frame=lines,framerule=2pt}

\newcommand{\newquote}[1]{``#1''}

\newcommand{\italic}[1]{\textit{#1}}

\newcommand{\fixheaderbadness}{
	\hbadness=10000 % A parameter that tells TeX at what point to report badness errors (i.e. underfull and overfull error). [number] ranges from 0 to 10000. 
	\vbadness=10000 % A parameter that tells TeX at what point to report badness errors (i.e. underfull and overfull error). [number] ranges from 0 to 10000. 
	\hfuzz=100pt % A parameter that allows hbox's to be overfull by [length] before an overfull error occurs. 
	\pretolerance=10000
}

\newcommand{\newchapter}[2]{
	
	\protect\chapter{#1}
	\label{#2}
	\setlength{\parskip}{1em}	
}

\newcommand{\centerItem}[1]{
	
	\begin{center}
		#1
	\end{center}
}

\newcommand{\imagefigurecaption}[4][0.35]{
	
	\begin{figure}[H]
		\centering
		\includegraphics[width=#1\textwidth,height=\textheight,keepaspectratio]{#2}
		\captionsetup{justification=centering}
		\caption{#3}
		\ifthenelse{\equal{#4}{}}{}{\label{#4}}
	\end{figure}	
}

\newcommand{\imagefigure}[3][0.35]{
	
	\imagefigurecaption[#1]{#2}{#3}{}	
}

\newcommand{\tikzimagefigure}[2]{
	
	\begin{figure}[H]
		\centering
		#1
		\captionsetup{justification=centering}
		\caption{#2}
	\end{figure}	
}

\newcommand{\tikzimagefigurelabel}[3]{
	
	\begin{figure}[H]
		\centering
		#2
		\captionsetup{justification=centering}
		\caption{#3}
		\ifthenelse{\equal{#1}{}}{}{\label{#1}}
	\end{figure}	
}

% https://tex.stackexchange.com/questions/37581/latex-figures-side-by-side
\newcommand{\asideimages}[8]{
	
	\begin{figure}[H]
		
		\centering
		\subfloat[#4]{
			{\includegraphics[width=#1]{#3}}
		}
		\qquad
		\subfloat[#6]{
			{\includegraphics[width=#2]{#5}}
		}
		\caption{#7}
		\label{#8}
		
	\end{figure}

}

\newenvironment{changemargin}[2]{%
	
	\begin{list}{}{%
			\setlength{\topsep}{3pt}%
			\setlength{\leftmargin}{#1}%
			\setlength{\rightmargin}{#2}%
			\setlength{\listparindent}{\parindent}%
			\setlength{\itemindent}{\parindent}%
			\setlength{\parsep}{\parskip}%
		}%
	\item[]}{\end{list}}
	
	
\newcommand{\centercaption}[1]{	
	\captionsetup{justification=centering}
	\protect\caption{#1}
}

\DeclareCaptionType{equ}[][]

\newcommand{\setequation}[5] {

	\begin{equ}[H]
		\begin{changemargin}{#1}{#2}
			\begin{figure}[H]
				#3
				\vspace{-0.5cm}
				\centercaption{#5}
				\label{#4}
			\end{figure}
		\end{changemargin}	
	\vspace{-1.5cm}
	\end{equ}
}

\newcommand{\centerformula}[1]{
	\vspace{-0.8cm}
	\begin{align}
		#1
	\end{align}
}

\newcommand{\mumax}{$mumax^3\ $}

\newcommand{\tpower}[1]{

	\ensuremath{\times\ 10^{#1}}
}

\newcommand{\super}[1]{\textsuperscript{#1}}

\newcommand{\sub}[1]{\textsubscript{#1}}

\newcommand{\formularefactor}[2]{
	\begin{equation}
		\vspace{-1.0cm}
		$$
			\centering{#1}
			\vspace{#2}%
		$$
	\end{equation}
}

% better diameter symbol
\DeclareFontEncoding{LS1}{}{}
\DeclareFontSubstitution{LS1}{stix}{m}{n}
\DeclareRobustCommand{\diameter}{%
	\text{\usefont{LS1}{stixscr}{m}{n}\symbol{"60}}%
}

\global\def\enableTODO{true}

\newcommand{\showtodos}[1] {
	\global\def\enableTODO{#1}
}

\definecolor{amber}{rgb}{1.0, 0.75, 0.0}

\newcommand{\TODO}[1]{
	\def\trueOption{true}
	\ifx\enableTODO\trueOption
		\ifx\printableimages\trueValue
			\todo[inline, bordercolor=black, color=white]{\bold{TODO}: #1}
		\else
			\todo[inline, bordercolor=black, color=red!50]{\bold{TODO}: #1}
		\fi
	\fi
}

\newcommand{\sticker}[1]{
	
	\ifx\printableimages\trueValue
		\todo[inline, bordercolor=black, color=white]{#1}
	\else
		\todo[inline, bordercolor=black, color=amber!50]{#1}
	\fi
}

% https://tex.stackexchange.com/questions/40561/table-with-multiple-lines-in-some-cells
\newcommand{\celltwolines}[2]{%

	\begin{tabular}{@{}c@{}}
		#1 \\ #2
	\end{tabular}
}

%% WATERMARK STUFF
%Customize watermark
\global\def\optionWatermark{true}

\newcommand{\docwatermark}[1]{
	
	\global\def\optionWatermark{#1}
	\newsavebox\mybox
	\savebox\mybox{\tikz[color=gray,opacity=0.3,font=\sffamily]\node{#1};}
	\newwatermark*[allpages,angle=45,scale=10,xpos=-30,ypos=15]{\usebox\mybox}
}

\newcommand{\showwatermark}[2]{
	
	\def\constPreliminary{PRELIMINARY}
	\def\truestr{true}
	\ignorespaces\lowercase{\def\tmp{#1}}\unskip
	\def\valueNull{\par}
	
	\ifx\tmp\valueNull
		\docwatermark{\constPreliminary}
		\global\def\optionWatermark{true}
	\else
		\ifx\tmp\truestr
			\ignorespaces\def\optValue{#2}\unskip
	
			\ifx\optValue\valueNull
				\docwatermark{\constPreliminary}
			\else
				\def\voidValue{}
				\ifx\optValue\voidValue
					\docwatermark{\constPreliminary}
				\else 
					\docwatermark{\uppercase{#2}}
				\fi
			\fi
	
			\global\def\optionWatermark{true}
		\else
			\global\def\optionWatermark{false}
		\fi
	\fi
}

\newcommand{\preliminary}[2]{
	
	\showwatermark{#1}{#2}
}

\newcommand{\showpreliminary}[2]{
	
	\showwatermark{#1}{#2}
}

\newcommand{\gothicletter}[2][black]{
	
	\yinipar{\color{#1}#2}\hspace{-0.5em}
}

\newcommand{\changeTitleDocument}[1]{
	
	\ignorespaces\lowercase{\def\tmp{#1}}\unskip
	
	\ifx\tmp\empty
		% value empty, nothing to do
	\else
		\global\def\THETITLE{#1}
	\end
}

% https://en.wikibooks.org/wiki/LaTeX/Customizing_Page_Headers_and_Footers
% remove work chapter from header
\pagestyle{fancy}
\fancypagestyle{phdfancy}{%
	\fancyhead{} % Clean headers
	\fancyhead[LE,RO]{\leftmark}
	\fancyhead[RE,LO]{\THETITLE}
	\fancyfoot[C]{\thepage}
	\renewcommand{\footrulewidth}{0.4pt}
	\renewcommand{\headrulewidth}{0.4pt}
	\renewcommand{\chaptermark}[1]{\markboth{\thechapter. {\slshape{##1}}}{}} 
}

% rework plain style to fancy
\fancypagestyle{plain}{

	\pagestyle{fancy}
	\fancypagestyle{phdfancyplain}{%
		\fancyhead{} % Clean headers
		\fancyfoot[C]{\thepage}
		\renewcommand{\footrulewidth}{0.4pt}
		\renewcommand{\headrulewidth}{0.0pt}
	}

	\pagestyle{phdfancyplain}
}

% no gap between page line and footnote
%\setlength{\footskip}{17pt}
% https://tex.stackexchange.com/questions/164367/how-to-make-footnotes-appear-at-bottom-of-the-footers-bar

%% START FOOTNOTE FOOTNOTE BELOW HORIZONTAL LINE
\makeatletter
\renewcommand\footnoterule{%
	\kern20\p@
	\hrule\@width\linewidth
	\kern2.6\p@}
\makeatother

\fancypagestyle{phdfancyfootnote}{%
	
	\pagestyle{fancy}
	\fancyhead{} % Clean headers
	\fancyhead[LE,RO]{\leftmark}
	\fancyhead[RE,LO]{\THETITLE}
	\fancyfoot[C]{\thepage}
	\renewcommand{\footrulewidth}{0.0pt}
	\renewcommand{\headrulewidth}{0.4pt}
	\renewcommand{\chaptermark}[1]{\markboth{\thechapter. {\slshape{##1}}}{}} 
}

%% END FOOTNOTE FOOTNOTE BELOW HORIZONTAL LINE

%% BETTER FOOTNOTE WITH FANCY STYLE
\newcommand\footnoteurl[2]{\thispagestyle{empty}\thispagestyle{phdfancyfootnote}\footnote{\href{#2}{#1}}}

\newenvironment{startpagenumbering}[2][0] {
	
%	\pagestyle{plain}

	\pagestyle{phdfancy}
%	\setlength{\footskip}{17pt}
	\pagenumbering{arabic}
	\setcounter{page}{#1}{
		#2
	}
}{}
