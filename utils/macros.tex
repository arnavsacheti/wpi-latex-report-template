% set paragraph spacing
\setlength{\parskip}{0.2em}

% renew commands

\renewcommand\thepage{\romannumeral\numexpr\value{page}-1\relax}

% rework glossary title
%\renewcommand{\glossarypreamble}{\vspace*{\baselineskip}\vspace*{\parskip}}

% rework glossary title (better)
\renewcommand{\glossarypreamble}{\vspace{-1.5cm}{\flushright\rule{\linewidth}{1pt}\vspace*{3ex}}\protect\thispagestyle{phdfancyempty}}

\renewcommand{\arraystretch}{1.2}

% glossary custom style: 
% https://tex.stackexchange.com/questions/327328/glossaries-style-change-add-units
\newglossarystyle{betterglossary}{%
	\setglossarystyle{list}%

	\renewcommand*{\glossentry}[2]{%

		\item[\glsentryitem{##1}\glstarget{##1}{\glossentryname{##1}\normalfont{:}}]

		\glossentrydesc{##1}. ##2
	}%

	\renewcommand*{\glsgroupskip}{}%
}

% Renew the glossary entry command to be always uppercase
% mfirstuc – Uppercase the first letter of a word
\renewcommand{\glsnamefont}[1]{\makefirstuc{#1}}

% REMOVE INDENTATION DESCRIPTION ENVIRONMENT GLOSSARY
% https://stackoverflow.com/questions/49843733/change-hanging-indentation-for-latex-description-environment
\makeatletter
\renewenvironment{description}%
{\list{}{\leftmargin=0pt % <------- Adjust this length
		\labelwidth\z@ \itemindent-\leftmargin
		\let\makelabel\descriptionlabel}}%
{\endlist}
\makeatother

%\renewcommand because amssymb package has a command called \bold
\renewcommand{\bold}[1]{\textbf{#1}}

% system modifications
 
\setlength{\heavyrulewidth}{0.07em}
\newcommand{\otoprule}{\midrule[\heavyrulewidth]}

\newdateformat{monthyeardate}{\monthname[\THEMONTH] \THEYEAR}
\newcommand{\monthtoday}{\date{\monthyeardate\today}}

% https://tex.stackexchange.com/questions/112932/today-month-as-text
\newdateformat{coverdate}{\monthname[\THEMONTH] \THEYEAR}

% new commands
\newcommand{\matrixbraket}[2]{
	$\begin{pmatrix}
	#1 \\
	#2 \\
	\end{pmatrix}$
}

\newminted{mumax3}{frame=lines,framerule=2pt}

\newcommand{\newquote}[1]{``#1''}

\newcommand{\italic}[1]{\textit{#1}}

\newcommand{\fixheaderbadness}{
	\hbadness=10000 % A parameter that tells TeX at what point to report badness errors (i.e. underfull and overfull error). [number] ranges from 0 to 10000. 
	\vbadness=10000 % A parameter that tells TeX at what point to report badness errors (i.e. underfull and overfull error). [number] ranges from 0 to 10000. 
	\hfuzz=100pt % A parameter that allows hbox's to be overfull by [length] before an overfull error occurs. 
	\pretolerance=10000
}

\newcommand{\newchapter}[2]{
	
	\protect\chapter{#1}
	\label{#2}
	\setlength{\parsep}{\parskip}%
}

\newcommand{\centerItem}[1]{
	
	\begin{center}
		#1
	\end{center}
}

% https://tex.stackexchange.com/questions/95273/tikz-overlay-png-or-pdf-image-over-another-pdf-figure
% https://tex.stackexchange.com/questions/368908/tikz-only-fill-75-of-the-node-background-with-color
% https://tex.stackexchange.com/questions/153334/change-the-baselinestretch-line-spacing-only-for-figure-captions
% Parameters:
% #1 --> name of main image
% #2 --> name of inset image
% #3 --> scale factor for main image
% #4 --> scale factor for inset image
% #5 --> placement coordinates for inset image
% #6 --> background colour in inset image
% #7 --> opacity for background colour in inset image
% #8 --> caption for figure
% #9 --> label for figure
\newcommand{\imageinsets}[9]{%
	\begin{figure}[H]
		\centering
		\begin{tikzpicture}[every node/.style={anchor=south west,inner sep=0pt}, x=1mm, y=1mm]
			\node (image1) at (0,0) {\includegraphics[width=#3\textwidth,height=\textheight,keepaspectratio]{#1}};
			\node[path picture={
				\fill[#6, opacity=#7](path picture bounding box.north west)rectangle
				($(path picture bounding box.north east)!1.0!(path picture bounding box.south east)$);
			}] (image2) at #5 {\includegraphics[width=#4\textwidth,height=\textheight,keepaspectratio]{#2}};
		\end{tikzpicture}%
		%			\captionsetup{justification=centering}
		%			\captionsetup{belowskip=\spaceBelowCaptionFigures, justification=raggedright, font={stretch=1.5}}
		\captionsetup{aboveskip=\spaceAboveCaptionSpace, belowskip=\spaceBelowCaptionFigures, justification=raggedright, singlelinecheck=true}
		\ifthenelse{\equal{#8}{}}{}{\caption{#8}}
		\ifthenelse{\equal{#9}{}}{}{\label{#9}}
	\end{figure}%
}%

\newcommand{\imageoverlay}[9]{\imageinsets{#1}{#2}{#3}{#4}{#5}{#6}{#7}{#8}{#9}}%

\newcommand{\imagefigurecaption}[4][0.35]{%
	\begin{figure}[H]
		\centering
		\includegraphics[width=#1\textwidth,height=\textheight,keepaspectratio]{#2}
		%		\captionsetup{justification=centering}
		\captionsetup{aboveskip=\spaceAboveCaptionSpace, belowskip=\spaceBelowCaptionFigures, justification=raggedright, singlelinecheck=true}
		\caption{#3}
		\ifthenelse{\equal{#4}{}}{}{\label{#4}}
	\end{figure}	
}
\newcommand{\imagefigure}[3][0.35]{
	
	\imagefigurecaption[#1]{#2}{#3}{}	
}

% https://tex.stackexchange.com/questions/37581/latex-figures-side-by-side
\newcommand{\asideimages}[9][false]{%
	\begin{figure}[H]
		% if captions from subfire are empty, remove the marker below the image
		\ifthenelse{\equal{#5}{}}{}{
			\ifthenelse{\equal{#7}{}}{}{
				%				\captionsetup[subfigure]{labelformat=empty}
				\captionsetup[subfigure]{belowskip=\spaceBelowCaptionSubfigures, labelformat=empty, justification=raggedright, singlelinecheck=true}
			}
		}
		\begin{subfigure}{.5\textwidth}
			\centering
			\includegraphics[width=#2]{#4}
			\ifthenelse{\equal{#5}{}}{}{\caption{#5}}
		\end{subfigure}%
		\begin{subfigure}{.48\textwidth}
			\centering
			\includegraphics[width=#3]{#6}
			\ifthenelse{\equal{#7}{}}{}{\caption{#7}}
		\end{subfigure}%
		\ignorespaces\lowercase{\def\tmpVal{#1}}
		%
		\ifthenelse{\equal{\tmpVal}{false}}{
			\centercaption{#8}
		}{	
			\nocitecentercaption{#8}
		}
		\label{#9}
	\end{figure}
}%

% used to avoid caption cite in LOF
\newcommand{\asideimagesnocite}[8]{
	\asideimages[true]{#1}{#2}{#3}{#4}{#5}{#6}{#7}{#8}
}

\newenvironment{changemargin}[2]{%
	
	\begin{list}{}{%
			\setlength{\topsep}{3pt}%
			\setlength{\leftmargin}{#1}%
			\setlength{\rightmargin}{#2}%
			\setlength{\listparindent}{\parindent}%
			\setlength{\itemindent}{\parindent}%
			\setlength{\parsep}{\parskip}%
		}%
		\item[]}{\end{list}}

\def\spaceBelowCaptionSubfigures{13pt}
\def\spaceBelowCaptionFigures{-13pt}
\def\spaceBelowCaptionEquations{-2pt}
\def\spaceAboveCaptionSpace{13pt}

\newcommand{\centercaptionequation}[1]{	
	%	\captionsetup{justification=centering}
	\captionsetup{aboveskip=\spaceAboveCaptionSpace, belowskip=\spaceBelowCaptionEquations, justification=raggedright, singlelinecheck=true}
	\protect\caption{#1}
}

\newcommand{\centercaption}[1]{	
	%	\captionsetup{justification=centering}
	\captionsetup{aboveskip=\spaceAboveCaptionSpace, belowskip=\spaceBelowCaptionFigures, justification=raggedright, singlelinecheck=true}
	\protect\caption{#1}
}

\newcommand{\nocitecentercaption}[1]{	
	%	\captionsetup{justification=centering}
	\captionsetup{aboveskip=\spaceAboveCaptionSpace, belowskip=\spaceBelowCaptionFigures, justification=raggedright, singlelinecheck=true}
	\protect\nocitecaption{#1}
}

%\newcommand{\centercaptionlisting}[1]{
%%	\captionsetup{justification=centering}
%	\captionsetup{belowskip=\spaceBelowCaptionFigures, justification=raggedright, singlelinecheck=true}
%	\protect\caption[#1]{\boldmath{#1}}
%}

\DeclareCaptionType{equ}[][]

% https://tex.stackexchange.com/questions/39315/how-to-change-the-numbering-for-different-figures
\newcommand{\equationname}{Equation}
\DeclareFloatingEnvironment[name={\protect\equationname}]{eqfigure}

% https://tex.stackexchange.com/questions/534403/how-to-format-custom-cwl-for-texstudio-completion
% https://tex.stackexchange.com/questions/596938/texstudio-newenvironment-not-recognized-as-math-mode-fix-with-cwl#596964
\newenvironment{addequation}[5] {%
	\edef\EnvParamThree{\unexpanded{#3}}%
	\edef\EnvParamFour{\unexpanded{#4}}%	
	\edef\EnvParamFive{\unexpanded{#5}}%	
	\begin{equ}[H]%
		% https://tex.stackexchange.com/questions/316646/renaming-the-label-of-figure-just-in-some-cases
		%		\renewcommand{\figurename}{Equation}
%		\vspace{-0.6cm}%
		\begin{changemargin}{#1}{#2}%
			\begin{eqfigure}[H]%	
			}{%
				\vspace{-0.7cm}%
				\centercaptionequation{\EnvParamFour}%
				\label{\EnvParamThree}%
				\ifthenelse{\equal{\EnvParamFive}{true}}{%
					\addequations{\EnvParamFour} % this is for equations listing
				}{}%
			\end{eqfigure}%
		\end{changemargin}%	
		\vspace{-1.5cm}%
	\end{equ}%
}%

\newcommand{\setequation}[5] {
	
	\begin{equ}[H]
		
		% https://tex.stackexchange.com/questions/316646/renaming-the-label-of-figure-just-in-some-cases
		%		\renewcommand{\figurename}{Equation}
		\begin{changemargin}{#1}{#2}
			\begin{eqfigure}[H]
				#3
				\vspace{-0.5cm}
				\centercaptionequation{#5}
				\label{#4}
				\addequations{#5} % this is for equations listing
			\end{eqfigure}
		\end{changemargin}	
		\vspace{-1.5cm}
	\end{equ}
}

\newenvironment{addequationboxed}[6] {%
	\edef\EnvParamThree{\unexpanded{#3}}%
	\edef\EnvParamFour{\unexpanded{#4}}%	
	\edef\EnvParamFive{\unexpanded{#5}}%	
	\edef\EnvParamSix{\unexpanded{#6}}%
	%
	\begin{equ}[H]%
		% https://tex.stackexchange.com/questions/316646/renaming-the-label-of-figure-just-in-some-cases
		%		\renewcommand{\figurename}{Equation}
		\begin{changemargin}{#1}{#2}%
			\begin{eqfigure}[H]%
			}{%
				\vspace{\EnvParamFive}%
				\centercaptionequation{\EnvParamFour}%
				%				\centercaptionformula{#5}
				\label{\EnvParamThree}%
				\ifthenelse{\equal{\EnvParamSix}{true}}{%
					\addequations{\EnvParamFour} % this is for equations listing
				}{}%
			\end{eqfigure}%
		\end{changemargin}%
		\vspace{-1.5cm}%
	\end{equ}%
}%

\newcommand{\setequationboxed}[6] {
	
	\begin{equ}[H]
		% https://tex.stackexchange.com/questions/316646/renaming-the-label-of-figure-just-in-some-cases
		%		\renewcommand{\figurename}{Equation}
		\begin{changemargin}{#1}{#2}
			\begin{eqfigure}[H]
				#3
				\vspace{#6}
				\centercaptionequation{#5}
				%				\centercaptionformula{#5}
				\label{#4}
				\addequations{#5} % this is for equations listing
			\end{eqfigure}
		\end{changemargin}
		\vspace{-1.5cm}
	\end{equ}
}

% https://tex.stackexchange.com/questions/436165/capture-the-content-within-an-environment
\NewEnviron{colorequbox}[2]{%
	\colorboxed{#1}{#2}{
		\BODY
	}
}

\newcommand{\centerformula}[1]{
	\vspace{-0.8cm}
	\begin{align}
		#1
	\end{align}
}

\newcommand{\tpower}[1]{
	\ensuremath{\times\ 10^{#1}}
}

\newcommand{\super}[1]{\textsuperscript{#1}}

\newcommand{\sub}[1]{\textsubscript{#1}}

% better diameter symbol
\DeclareFontEncoding{LS1}{}{}
\DeclareFontSubstitution{LS1}{stix}{m}{n}
\DeclareRobustCommand{\diameter}{%
	\text{\usefont{LS1}{stixscr}{m}{n}\symbol{"60}}%
}

\global\def\enableTODO{true}

\newcommand{\showtodos}[1] {
	\global\def\enableTODO{#1}
}

\definecolor{amber}{rgb}{1.0, 0.75, 0.0}

\definecolor{red-clean}{rgb}{1.0, 0.41, 0.38}

\definecolor{red}{rgb}{1,0,0}

\definecolor{amber}{rgb}{1.0, 0.75, 0.0}

\definecolor{red-light}{rgb}{1.0, 0.41, 0.38}

\definecolor{blue-light}{rgb}{0.58, 0.86, 1}

\newcommand{\TODO}[1]{
	\def\trueOption{true}
	\ifx\enableTODO\trueOption
		\ifx\printableimages\trueValue
			\todo[inline, shadow=0.8, bordercolor=black, color=white]{\qquad\bold{TODO}: #1}
		\else
			\todo[inline, shadow=0.8, bordercolor=black, color=red!50]{\qquad\bold{TODO}: #1}
		\fi
	\fi
}

\makeatletter
\newcommandtwoopt{\sticker}[3][][true]{%
	
	\ignorespaces\lowercase{\def\valueTmp{#1}}
	\ignorespaces\lowercase{\def\indentStatusParagraph{#2}}

	\def\itemWidth{long}
	
	\ifx\printableimages\trueValue
	
		\ifx\valueTmp\itemWidth
			\todo[inline, shadow=0.8, bordercolor=black, color=white]{\qquad#3}
		\else
			\todo[inline, shadow=0.8, bordercolor=black, color=white]{\qquad\centerline{#3}}
		\fi
		
	\else
	
		\ifx\valueTmp\itemWidth
			\todo[inline, shadow=0.8, bordercolor=black, color=amber!50]{\qquad#3}
		\else
			\todo[inline, shadow=0.8,bordercolor=black, color=amber!50]{\qquad\centerline{#3}}
		\fi
		
	\fi
	
	\ifx\indentStatusParagraph\trueValue
		\par\@afterindentfalse\@afterheading % avoid indent on first paragraph after TODO note
	\fi
	% https://tex.stackexchange.com/questions/229001/do-not-indent-second-paragraph-after-macro-was-used
}
\makeatother


\makeatletter

\newcommandtwoopt{\stickerMultiline}[3][amber!50][true] {

		% Set true if TODO note is right below the title else set false on #2
		\ignorespaces\lowercase{\def\valueTmp{#2}}
		\ignorespaces\lowercase{\def\valueBackColourTmp{#1}}
		
		\ifx\printableimages\trueValue
			\ignorespaces\lowercase{\def\valueBackColourTmp{white}}
		\fi 

		\if\valueTmp\trueValue
		
			\todo[inline, shadow=0.8, bordercolor=black, color=\valueBackColourTmp]{#3}
			% check if user sets the TODO right below title or not
			\par\@afterindentfalse\@afterheading % avoid indent on first paragraph after TODO note
			
		\else
		
			\@nobreaktrue
			\todo[inline, shadow=0.8, bordercolor=black, color=\valueBackColourTmp]{#3}
			\@nobreakfalse
			
		\fi
}

\makeatother

% https://tex.stackexchange.com/questions/40561/table-with-multiple-lines-in-some-cells
\newcommand{\celltwolines}[2]{%

	\begin{tabular}{@{}c@{}}
		#1 \\ #2
	\end{tabular}
}

%% WATERMARK STUFF
% Customize watermark
%\global\def\optionWatermark{true}
%
%\newcommand{\docwatermark}[1]{
%	
%	\global\def\optionWatermark{#1}
%	\newsavebox\mybox
%	\savebox\mybox{\tikz[color=gray,opacity=0.3,font=\sffamily]\node{#1};}
%	\newwatermark*[allpages,angle=45,scale=10,xpos=-30,ypos=15]{\usebox\mybox}
%}
%
%\newcommand{\showwatermark}[2]{
%	
%	\def\constPreliminary{PRELIMINARY}
%	\def\truestr{true}
%	\ignorespaces\lowercase{\def\tmp{#1}}\unskip
%	\def\valueNull{\par}
%	
%	\ifx\tmp\valueNull
%		\docwatermark{\constPreliminary}
%		\global\def\optionWatermark{true}
%	\else
%		\ifx\tmp\truestr
%			\ignorespaces\def\optValue{#2}\unskip
%	
%			\ifx\optValue\valueNull
%				\docwatermark{\constPreliminary}
%			\else
%				\def\voidValue{}
%				\ifx\optValue\voidValue
%					\docwatermark{\constPreliminary}
%				\else 
%					\docwatermark{\uppercase{#2}}
%				\fi
%			\fi
%	
%			\global\def\optionWatermark{true}
%		\else
%			\global\def\optionWatermark{false}
%		\fi
%	\fi
%}
%
%\newcommand{\preliminary}[2]{
%	
%	\showwatermark{#1}{#2}
%}
%
%\newcommand{\showpreliminary}[2]{
%	
%	\showwatermark{#1}{#2}
%}
%
%% END WATERMARK STUFF

\newcommand{\gothicletter}[2][black]{
	
	\yinipar{\color{#1}#2}\hspace{-0.5em}
}

\newcommand{\changeTitleDocument}[1]{
	
	\ignorespaces\lowercase{\def\tmp{#1}}\unskip
	
	\ifx\tmp\empty
		% value empty, nothing to do
	\else
		\global\def\THETITLE{#1}
	\end
}

% rework toc, lot and lof title, add line below
\renewcommand{\cftaftertoctitle}{\\\rule{\linewidth}{1pt}\vspace*{3ex}}
\renewcommand{\cftafterlottitle}{\\\rule{\linewidth}{1pt}\vspace*{3ex}}
\renewcommand{\cftafterloftitle}{\\\rule{\linewidth}{1pt}\vspace*{3ex}}

% rework bibliography title
% https://tex.stackexchange.com/questions/70025/titlesec-and-bibliography
% https://latex.org/forum/viewtopic.php?t=28709
\newcommand{\SetTitleBibliography}[1]{
	\renewcommand{\bibname}{#1}
}

\newcommand{\titlebibliography}[1]{
	\addto{\captionsenglish}{\bibname}
}

\def\customquad{\hskip\baselineskip\relax}

\patchcmd{\thebibliography}{\chapter*{\bibname}}{\chapter*{\flushleft\bibname\vspace{-0.6cm}\customquad\rule{\linewidth}{1pt}\vspace{-0.5cm}}\protect{\thispagestyle{phdfancyempty}}}{}{}

% redefine plain style to empty style
% https://tex.stackexchange.com/questions/65089/im-trying-to-redefine-the-plain-pagestyle-as-empty
\makeatletter
\let\ps@plain\ps@empty
\makeatother

% create new chapter without page reference in TOC, optional, to set an horizontal line below the title
% create new chapter without page reference in TOC, optional, to set an horizontal line below the title
\newcommandtwoopt{\chaptercustompageintoc}[4][showrule][true]{
	
	% https://tex.stackexchange.com/questions/50038/suppress-page-numbers-when-using-addcontentsline
	\ifthenelse{\equal{#1}{showrule}}{
		
		\chapter*{\flushleft#3\vspace{-0.5cm}\customquad\rule{\linewidth}{1pt}\vspace{-1.2cm}}
		
	}{
		
		\chapter*{\flushleft{#3}\vspace{-0.5m}\customquad}
	}
	
	\ifthenelse{\equal{#2}{true}}{
		\cftaddtitleline{toc}{chapter}{#3}{#4}
	}
}


% plain pagestyle
\newcommandtwoopt{\chapterpagenumberintoc}[3][showrule][true] {
	
	\ifthenelse{\equal{#1}{showrule}}{
		
		\chapter*{\flushleft#3\vspace{-0.6cm}\customquad\rule{\linewidth}{1pt}\vspace{-1.2cm}}
	}{
		
		\chapter*{\flushleft{#3}\vspace{-0.6cm}\customquad}
	}
	
	\ifthenelse{\equal{#2}{true}}{
		\addcontentsline{toc}{chapter}{#3}
	}
}

\newcommand{\chaptercustompageintoctikz}[4][true]{
	
	\DeclareRobustCommand{\chapnumfontcustomappendix}{\fontseries{bx}\fontsize{30}{35}\selectfont}%
	\DeclareRobustCommand{\chaptitlefontcustomappendix}{\bfseries\fontsize{20}{25}\selectfont}%
	
	% https://tex.stackexchange.com/questions/50038/suppress-page-numbers-when-using-addcontentsline
	%	\ifthenelse{\equal{#1}{showrule}}{%
	%
	\chapter*{\filright\chapnumfontcustomappendix\flushleft{\MakeUppercase{#2}}\\%
		\vspace{1ex}\titlerule%
		\flushright\chaptitlefontcustomappendix#3\\%
		\vspace{1.5ex}\titlerule\vspace{1ex}\vspace{-1.2cm}}%
	%
	%	}{%
	%		\chapter*{\flushleft{#3\ -\ #4}\vspace{-0.5m}\customquad}%
	%		\relax
	%	}%
	
	\ifthenelse{\equal{#1}{true}}{
		\cftaddtitleline{toc}{chapter}{#2~-~#3}{\protect{#4}}
	}
}

%% input file for abstract

\global\def\showpageabstract{false}
\global\def\showrule{showrule}
\global\def\showintoc{true}

\newcommandtwoopt{\inputabstract}[4][false][showrule]{
	
	\ignorespaces\lowercase{\def\tmp{#1}}
	\global\def\showpageabstract{\tmp}
	
	\ignorespaces\lowercase{\def\tmprule{#2}}
	\global\def\showrule{\tmprule}
	
	% whether or not to show in TOC
	\ignorespaces\lowercase{\def\tmptoc{#3}}
	\global\def\showintoc{\tmptoc}
	
	\ifthenelse{\equal{\tmp}{false}}{
		
		% abstract without page numbers and without page number in TOC
		\protect\pagestyle{empty}\input{#4}
		\protect\thispagestyle{empty}
	}{
		
		% abstract with page numers and page number in TOC
		\protect\pagestyle{phdfancyempty}\input{#4}
		\protect\thispagestyle{phdfancyempty}
	}
	
	\clearpage
}

\newcommand{\abstractname}{Abstract}

\newcommand{\inputabstracttitle}{
	
	\ifthenelse{\equal{\showpageabstract}{false}}{

		\chaptercustompageintoc[\showrule][\showintoc]{\abstractname}{}		
	}{
	
		\chapterpagenumberintoc[\showrule][\showintoc]{\abstractname}
	}
}

%% end input file for abstract

% environment for justified text
\newenvironment{justified}[1] {
	\setlength\parindent{0pt}
	\setlength{\parsep}{\parskip}%
	#1
}{}

% environment italic text
\newenvironment{shapeitalic} {
	\itshape
}{}

%% START Create a list of equations

% https://tex.stackexchange.com/questions/270611/creating-list-of-equations
\newcommand{\equationpagetext}{Page}%
\newcommand{\listequationsname}{List of Equations}%
\newcommand{\listequationtitle}{\vspace{-0.4cm}\listequationsname\vskip0cm\vspace{-0.59cm}\rule{\linewidth}{1pt}\vspace{1.8cm}\\\protect{\hspace*{0pt}\hfill\fontsize{11}{12}\textbf{\equationpagetext}\par\protect\thispagestyle{phdfancyempty}\vspace{-1.0cm}}}%
\newlistof[figure]{equations}{equ}{\listequationtitle}%
\newcommand{\addequations}[1]{%
	%	\setlength{\cftequationsindent}{0.5em}%
	%	\setlength{\cftequationsnumwidth}{1.3em}%
	%	\setlength\cftequationsnumwidth{10em}%
	\addcontentsline{equ}{figure}{\protect\normalfont\numberline{\theeqfigure}\protect\normalfont#1}\par%
}%

%% END Create a list of equations

% change default color to superscript footnote
% https://tex.stackexchange.com/questions/26693/change-the-color-of-footnote-marker-in-latex
\makeatletter

\ifdefined\compileblack
	\def\footnotecolour{\color{black}}
\else
	\def\footnotecolour{\color{red-clean}}
\fi

\renewcommand\@makefnmark{\hbox{\@textsuperscript{\normalfont\footnotecolour\@thefnmark}}}
\renewcommand\@makefntext[1]{%
	\parindent 1em\noindent
	\hb@xt@1.8em{%
		\hss\@textsuperscript{\normalfont\@thefnmark}}#1}
\makeatother

%% START List of Publications 

% USE multibib package: https://www.overleaf.com/learn/latex/Multibib
% https://tex.stackexchange.com/questions/170316/nocite-for-single-bibdatasources-with-biblatex-biber
% https://www.privacypies.org/blog/2017/01/25/bib-management.html
\newcommand{\listofpublicationsname}{List of Publications}

%% END List of Publications

\newcommand{\tikzimagefigure}[2]{

	\begin{figure}[H]
		\centering
		#1
%		\captionsetup{justification=centering}
		\captionsetup{belowskip=\spaceBelowCaptionFigures, justification=raggedright, singlelinecheck=true}
		\caption{#2}
	\end{figure}
}

\newcommand{\tikzimagefigurelabel}[3]{

	\begin{figure}[H]
		\centering
		#2
%		\captionsetup{justification=centering}
		\captionsetup{belowskip=\spaceBelowCaptionFigures, justification=raggedright, singlelinecheck=true}
		\caption{#3}
		\ifthenelse{\equal{#1}{}}{}{\label{#1}}
	\end{figure}
}

% Change the page numbering style and continue with the page counter
% https://tex.stackexchange.com/questions/453254/continue-previous-page-numbering-roman-arabic-with-two-independent-counters
% #1 -> Choose the numbering style: Arabic, Roman, etc.
% #2 -> Choose variable holding the page number, default buit-in variable named "page"
\newcommandtwoopt{\ContinuePaginingWithNumbering}[2][arabic][page] {

	\newcounter{savepagetoken}
	\setcounter{savepagetoken}{\value{#2}}
	\pagenumbering{#1}
	\setcounter{#2}{\value{savepagetoken}}
}

% change colour of glossary links
% https://tex.stackexchange.com/questions/38544/glossary-links-color
%\def\glossarylinknewcolour{orange}
%
%\makeatletter
%\newcommand{\glsplainhyperlink}[2]{\colorlet{currenttext}{\glossarylinknewcolour}\colorlet{currentlink}{\@linkcolor}\hypersetup{linkcolor=currenttext}\hyperlink{#1}{#2}\hypersetup{linkcolor=currentlink}}
%\let\@glslink\glsplainhyperlink
%\makeatother

% Fix space TOC/LOT/LOF
% https://tex.stackexchange.com/questions/69321/spacing-between-chapters-in-list-of-tables
\makeatletter

\def\@chapter[#1]#2 {
	
	\ifnum\c@secnumdepth > \m@ne
		\if@mainmatter
			\refstepcounter{chapter}%
			\typeout{\@chapapp\space\thechapter.}%
			\addcontentsline{toc}{chapter}{\protect\numberline{\thechapter}#1}%
		\else
			\addcontentsline{toc}{chapter}{#1}%
		\fi
	\else
		\addcontentsline{toc}{chapter}{#1}%
	\fi
	
	\chaptermark{#1}%
	%                    \addtocontents{lof}{\protect\addvspace{10\p@}}% NEW
	%                    \addtocontents{lot}{\protect\addvspace{10\p@}}% NEW
	\if@twocolumn
		\@topnewpage[\@makechapterhead{#2}]%
	\else
		\@makechapterhead{#2}%
		\@afterheading
	\fi
}

\makeatother

% Make subsubsubsections available
% https://tex.stackexchange.com/questions/60209/how-to-add-an-extra-level-of-sections-with-headings-below-subsubsection
% https://tex.stackexchange.com/questions/186981/is-there-a-subsubsubsection-command
\newif\ifexpandsubsublevel

\expandsubsublevelfalse

\ifexpandsubsublevel
	\newcommand{\subsubsubsection}[1]{\paragraph{#1}\mbox{}\\}
\fi

%% Avoid cite in captions for LOF/LOT
%% https://tex.stackexchange.com/questions/147936/list-of-figures-citation-issue
\DeclareRobustCommand\nocitecaptionlink[1]{{\def\cite##1{}#1}}
\newcommand\nocitecaption[1]{\caption[\nocitecaptionlink{#1}]{#1}}

% Bibliography space patching separation between number and text 
% https://stackoverflow.com/questions/26203769/right-aligned-reference-labels-with-multibib
\makeatletter
\patchcmd{\thebibliography}{%
	\section*{\refname}\@mkboth{\MakeUppercase\refname}{\MakeUppercase\refname}%
}{}{}{}
\begingroup\catcode`#=12
\AtBeginDocument{
	\patchcmd\thebibliography
	{\advance\@tempcnta#1}
	{\advance\@tempcnta#1\else\@tempcnta#1}
	{}{}
}
\endgroup
\makeatother

%% Colour boxing equations

\makeatletter
\renewcommand{\boxed}[2][\fboxsep]{{%
		\setlength{\fboxsep}{#1}\fbox{\m@th$\displaystyle#2$}}}
\makeatother

% https://tex.stackexchange.com/questions/326375/boxed-border-color
% Syntax: \colorboxed[<color model>]{<color specification>}{<math formula>}

\newcommand*{\colorboxed}{}
\def\colorboxed#1#{%
	\colorboxedAux{#1}%
}

\newcommand{\colorboxedAux}[4]{%
	% #1: optional argument for color model
	% #3: color specification
	% #4: formula
	\begingroup
	\colorlet{cb@saved}{.}%
	\color#1{#2}%
	\begin{center}
		\boxed[#3]{%
			\color{cb@saved}%
			#4%
		}%
	\end{center}
	\endgroup
}

%% END Colour boxing equations

% FIX SPACE TOC FOR LONG TITLE
% https://tex.stackexchange.com/questions/283730/left-align-toc-items-when-using-tableofcontents
\makeatletter
\bgroup
\advance\@flushglue by \@tocrmarg
\xdef\@tocrmarg{\the\@flushglue}%
\egroup
\makeatother

% figure with four images
\newcommand{\squarefigure}[9][0.35]{%
	\edef\tempa{\unexpanded{#1}}%
	\edef\tempb{\unexpanded{#2}}%
	\edef\tempc{\unexpanded{#3}}%
	\edef\tempd{\unexpanded{#4}}%
	\edef\tempe{\unexpanded{#5}}%
	\edef\tempf{\unexpanded{#6}}%
	\edef\tempg{\unexpanded{#7}}%
	\edef\temph{\unexpanded{#8}}%
	\squarefigurecontinue{\unexpanded{#9}}%
}

\newcommand{\squarefigurecontinue}[7]{
	\begin{figure}[H]
		\centering
		\begin{subfigure}[c]{.475\linewidth}
			\centering
			\includegraphics[width=\tempa\textwidth,height=\textheight,keepaspectratio]{\tempb}
			\ifthenelse{\equal{\tempf}{}}{}{\caption{\tempf}}%
			\ifthenelse{\equal{#4}{}}{}{\label{#4}}%
		\end{subfigure}\hspace*{0.02cm}%\hfill % <-- "\hfill"
		\begin{subfigure}[c]{.475\linewidth}
			\centering
			\includegraphics[width=\tempa\textwidth,height=\textheight,keepaspectratio]{\tempc}
			\ifthenelse{\equal{\tempg}{}}{}{\caption{\tempg}}%
			\ifthenelse{\equal{#5}{}}{}{\label{#5}}%
		\end{subfigure}
		
		\medskip\medskip% create some *vertical* separation between the graphs
		\begin{subfigure}[c]{.475\linewidth}
			\centering
			\includegraphics[width=\tempa\textwidth,height=\textheight,keepaspectratio]{\tempd}
			\ifthenelse{\equal{\temph}{}}{}{\caption{\temph}}%
			\ifthenelse{\equal{#6}{}}{}{\label{#6}}%
		\end{subfigure}\hspace*{0.02cm}%\hfill % <-- "\hfill"
		\begin{subfigure}[c]{.475\linewidth}
			\centering
			\includegraphics[width=\tempa\textwidth,height=\textheight,keepaspectratio]{\tempe}
			\ifthenelse{\equal{#1}{}}{}{\caption{#1}}%
			\ifthenelse{\equal{#7}{}}{}{\label{#7}}%
		\end{subfigure}
		\captionsetup{belowskip=\spaceBelowCaptionFigures, justification=justified, singlelinecheck=true}%
		\ifthenelse{\equal{#2}{}}{}{\caption{#2}}%
		\label{#3}
	\end{figure}
}%

% https://tex.stackexchange.com/questions/11668/adding-unnumbered-sections-to-toc
\newcommand*{\unnumberedsection}[2]{%
	\section*{#1}\addcontentsline{toc}{subsection}{#1}%
	\label[subsection]{#2}%
}%

% https://tex.stackexchange.com/questions/102529/reference-to-section-where-is-label
\newcommand*{\customnameref}[2]{\protect\hyperref[#1]{#2}}%

\newcommand{\forcestylebackmatter}{%
	%https://tex.stackexchange.com/questions/336399/page-number-font-doesnt-change-on-toc-lof-and-lot-pages
	\addtocontents{toc}{\protect\renewcommand{\protect\cftsecfont}{\normalfont\itshape}}%
	\addtocontents{toc}{\protect\renewcommand{\protect\cftsecpagefont}{\normalfont\itshape}}%
}%

% https://tex.stackexchange.com/questions/117474/splitting-toc-into-two-parts-each-in-one-page
\newcommand{\tocclearpage}{\addtocontents{toc}{\protect\clearpage}}

% https://tex.stackexchange.com/questions/313279/set-counter-chapter-with-float
\newcommand{\setcounterappendix}[3]{%
	\setcounter{chapter}{#1}%
	\ifthenelse{\equal{#2}{true}}{%
		\ifthenelse{\equal{#3}{roman}}{%
			\renewcommand{\thechapter}{\Roman{chapter}}%
		}{%
			\renewcommand{\thechapter}{\arabic{chapter}}%
		}%
		\renewcommand{\thesection}{\thechapter.\Alph{section}}%
		%		\renewcommand*{\thesubsection}{\thesection.\Alph{subsection}}
	}{}%
}%


% Check if counter exists
% https://tex.stackexchange.com/questions/155776/check-if-counter-exists
\makeatletter
\newcommand*\ifcounter[1]{%
	\ifcsname c@#1\endcsname%
	\expandafter\@firstoftwo%
	\else%
	\expandafter\@secondoftwo%
	\fi%
}%
\makeatother

% Create new counter
\newcommand{\checkorcreatecounter}[1]{\ifcounter{#1}{\relax}{\newcounter{#1}}}%

% only for overleaf in local installation, solve glossary extra space on the left
%\let\oldgls\gls
%\renewcommand{\gls}[1]{\hspace{-1.9em}\oldgls{#1}}

%%% Affirmation
\def\labelDeclaration{Declaration}%

\def\declarationText{I hereby declare that the work presented in this thesis is entirely my own and that I did not use any other sources and references than the listed ones. I have marked all direct or indirect statements from other sources contained therein as quotations. Neither this work nor significant parts of it were part of another examination procedure. I have not published this work in whole or in part before. The electronic copy is consistent with all submitted copies.
}

\def\labelSignature{\ Zaragoza (Aragón), \coverdate\today}

\newcommand{\declaration} {
	\cleardoublepage
	\null
	\vskip 5cm\relax
	\begin{center}
		\begin{minipage}[t]{10cm}
			\hbox{\textbf{\labelDeclaration}}%
			\vskip 1cm\relax
			\declarationText
			\vskip 4cm\relax
			\hrulefill
			\vskip .4\baselineskip
			\hbox{\labelSignature}
		\end{minipage}
	\end{center}
}

\newcommand{\inputdeclaration}[1][false] {

	\ignorespaces\lowercase{\def\tmp{#1}}
	
	\ifthenelse{\equal{\tmp}{false}}{
	
		% declaration without page numbers and without page number in TOC
		\protect\pagestyle{empty}\phantomsection\declaration
		\cftaddtitleline{toc}{chapter}{\labelDeclaration}{}
		\protect\thispagestyle{empty}
	}{
		
		% declaration with page numers and page number in TOC	
		\protect\pagestyle{phdfancyempty}\phantomsection\declaration
		\addcontentsline{toc}{chapter}{\labelDeclaration}
		\protect\thispagestyle{phdfancyempty}
	}

	\clearpage
}

%%% Different chapter title styles used in utils/fance_styles
%%%%%%%%%%%%%%%%%%%%%%%%%%%%%%%%%%%%%%%%%%%%%%%%%%%
%%            REWORK TITLE CHAPTERS              %%
%%%%%%%%%%%%%%%%%%%%%%%%%%%%%%%%%%%%%%%%%%%%%%%%%%%
\newcommand{\hsp}{\hspace{0pt}}
\newcommand{\chapnumfont}{\bfseries\fontsize{30}{35}\selectfont}%
\newcommand{\chaptitlefont}{\fontseries{b}\fontsize{30}{35}\selectfont}%
\newcommand{\chapnumfontcustom}{\fontseries{b}\scshape\fontsize{30}{35}\selectfont}%
\newcommand{\chaptitlefontcustom}{\bfseries\scshape\fontsize{35}{40}\selectfont}%

\definecolor{gray75}{gray}{0.75}%

\DeclareRobustCommand{\ChapterTitleStyleOne}{
	%%%%%%%%%%%%%%%%%%%%%%%%%%%%%%%%%%%%%%%%%%%%%%%%%%%
	%
	%% https://tex.stackexchange.com/questions/310413/optimize-chapter-styling 
	%	\definecolor{gray75}{gray}{0.75}	
	%	\titleformat{\chapter}[hang]{\vspace{-4.8cm}\flushright%
	%		\fontseries{b}\fontsize{80}{100}\selectfont}{\fontseries{b}\fontsize{100}{130}\selectfont \textcolor{gray75}\thechapter\hsp}{0pt}{\\ \vspace{-1cm}\Huge\bfseries}[\titlerule\vspace{-1cm}]%
	%	
	\titleformat{\chapter}[hang]{\vspace{-4.8cm}\flushright%
		\fontseries{b}\fontsize{80}{100}\selectfont}{\fontseries{b}\fontsize{100}{130}\selectfont \textcolor{gray75}\thechapter\hsp}{0pt}{\\ \vspace{-1.2cm}\Huge\bfseries}[\titlerule\vspace{-1cm}]
	%
	%%%%%%%%%%%%%%%%%%%%%%%%%%%%%%%%%%%%%%%%%%%%%%%%%%%
}

\DeclareRobustCommand{\ChapterTitleStyleTwo}{
	%%%%%%%%%%%%%%%%%%%%%%%%%%%%%%%%%%%%%%%%%%%%%%%%%%
	%
	% https://tex.stackexchange.com/questions/68745/possible-values-for-fontseries-and-fontshape
	\titleformat{\chapter}[display]{%
		\Large\filcenter%
	}{%
		\titlerule[0.6ex]\vspace{0.1cm}\titlerule[0.6pt]\vspace{0.5ex}\bfseries%
		\chapnumfontcustom\filcenter\chaptername~\thechapter\\%
		\titlerule[0.6pt]%
	}{-0.1ex}{
		\vspace{-1.0cm}\chaptitlefontcustom\hsp\vspace{-1.3cm}
	}%
	\titlespacing*{\chapter}{0pt}{-32pt}{48pt}%
	%
	%%%%%%%%%%%%%%%%%%%%%%%%%%%%%%%%%%%%%%%%%%%%%%%%%%
}

\DeclareRobustCommand{\ChapterTitleStyleThree}{
	%%%%%%%%%%%%%%%%%%%%%%%%%%%%%%%%%%%%%%%%%%%%%%%%%%
	%
	% https://tex.stackexchange.com/questions/403898/recreating-the-fancy-chapter-style-of-a-textbook
	\newcommand{\chapnumfontSpecial}{%
		\fontsize{100}{0}%
		\selectfont%
	}%
	
	\colorlet{chapnumcol}{black!55}%
	
	\titleformat{\chapter}[display]%
	{\bfseries}%
	{\begin{tikzpicture}%
			\node[minimum width=\textwidth, text=black!25, fill=black!25, inner sep=1, outer sep=0, anchor=south] (rectinit) {\huge CHAPTER};
			\node[minimum width=.8\textwidth, text=white, inner sep=1, outer sep=0, anchor=south west, text width=.8\textwidth, align=right] at (rectinit.south west) (chapname) {\huge CHAPTER~~};
			\node[minimum width=.2\textwidth, inner sep=0, outer sep=0, anchor=south west, text width=.2\textwidth, align=left] at (chapname.south east) {\chapnumfontSpecial\textcolor{chapnumcol}{\thechapter}};
	\end{tikzpicture}}%
	{0pt}%
	{\vspace{-0.4cm}\chaptitlefontcustom\hsp\vspace{-1.5cm}}%
	\titlespacing*{\chapter}{0pt}{-32pt}{48pt}%
	%
	%%%%%%%%%%%%%%%%%%%%%%%%%%%%%%%%%%%%%%%%%%%%%%%%%%
}

\DeclareRobustCommand{\ChapterTitleStyleFour}{
	%%%%%%%%%%%%%%%%%%%%%%%%%%%%%%%%%%%%%%%%%%%%%%%%%%
	%
	% https://tex.stackexchange.com/questions/530234/how-to-customize-chapter-heading-in-a-fancy-way
	\colorlet{rulecolor}{black}
	\titleformat{\chapter}[display]
	{\filcenter}{\mbox{}\xrfill[0.4ex]{3pt}[rulecolor]\textsc{\large\bfseries\enspace\chaptername \thechapter}\enspace\xrfill[0.4ex]{3pt}[rulecolor]\mbox{}}{0.3ex} {{\color{rulecolor}\titlerule[1pt]}\vskip1.0ex\chaptitlefontcustom}[\medskip{\vskip-0.4ex\color{rulecolor}\titlerule[1pt]\vspace{-1.0cm}}]	
	\titlespacing*{\chapter}{0pt}{-32pt}{48pt}%
	%
	%%%%%%%%%%%%%%%%%%%%%%%%%%%%%%%%%%%%%%%%%%%%%%%%%%%
}


\DeclareRobustCommand{\ChapterTitleStyleFive}{
	%%%%%%%%%%%%%%%%%%%%%%%%%%%%%%%%%%%%%%%%%%%%%%%%%%%
	%
	% https://tex.stackexchange.com/questions/163052/customizing-chapters-title
	\newcommand{\chaptitlefontcustomOther}{\bfseries\scshape\fontsize{35}{40}\selectfont}%
	
	\titleformat{\chapter}[block]
	{\filright}
	{\parbox{\widthof{\Huge\sffamily\MakeUppercase{\chaptername}}}{%
			\filcenter\Large\usefont{T1}{phv}{m}{n}\Huge\MakeUppercase{\chaptername}\vskip -0.6cm%
			\fontsize{90pt}{90pt}\selectfont\thechapter%
		}\enspace%
	}
	{15pt}
	{\chaptitlefontcustomOther\usefont{T1}{phv}{b}{n}\vrule width 2pt \enspace%
		\parbox{\textwidth-\widthof{\LARGE\sffamily\MakeUppercase{\chaptername}\enspace}}{\filright}%
		\vspace*{-1.0cm}%
	}
	\titlespacing*{\chapter}{0pt}{-32pt}{48pt}%
	%
	%%%%%%%%%%%%%%%%%%%%%%%%%%%%%%%%%%%%%%%%%%%%%%%%%%%
}

\def\SetChapterTitleStyle#1{%
	\IfEqCase{#1}{%
		{1}{\protect\ChapterTitleStyleOne\relax}%
		{2}{\protect\ChapterTitleStyleTwo\relax}%
		{3}{\protect\ChapterTitleStyleThree\relax}%
		{4}{\protect\ChapterTitleStyleFour\relax}%
		{5}{\protect\ChapterTitleStyleFive\relax}%
	}[\protect\ChapterTitleStyleOne\relax]%
}%

%%% FancyHDR page style
% https://en.wikibooks.org/wiki/LaTeX/Customizing_Page_Headers_and_Footers
% remove work chapter from header

\fancypagestyle{phdfancy}{%
	
	\pagestyle{fancy}
	\fancyhead{} % Clean headers
	\fancyhead[LE,RO]{\THETITLE}
	\fancyhead[RE,LO]{\leftmark}
	\fancyfoot[C]{\thepage}
	\renewcommand{\footrulewidth}{0.4pt}
	\renewcommand{\headrulewidth}{0.4pt}
	\renewcommand{\chaptermark}[1]{\markboth{\thechapter. {\slshape{##1}}}{}} 
}

\newcommand{\SetHeaderTitle}[1]{
	\global\def\contentHeaderTitle{#1}
}

\fancypagestyle{phdfancyspecialempty}{%
	
	\pagestyle{fancy}
	\fancyhead{} % Clean headers
	\fancyhead[LE,RO]{\thepage}
	\fancyhead[RE,LO]{\contentHeaderTitle}
	\fancyfoot[C]{}
	\renewcommand{\footrulewidth}{0.0pt}
	\renewcommand{\headrulewidth}{0.4pt}
}

\fancypagestyle{phdfancyempty}{%
	
	\pagestyle{fancy}
	\fancyhead{} % Clean headers
	\fancyhead[LE,RO]{}
	\fancyhead[RE,LO]{}
	\fancyfoot[C]{\thepage}
	\renewcommand{\footrulewidth}{0.0pt}
	\renewcommand{\headrulewidth}{0.0pt}
}

\fancypagestyle{phdplainfancy}{%
	
	\pagestyle{fancy}
	\fancyhead{} % Clean headers
	\fancyhead[LE,RO]{}
	\fancyhead[RE,LO]{}
	\fancyfoot[C]{\thepage}
	\renewcommand{\footrulewidth}{0.4pt}
	\renewcommand{\headrulewidth}{0.0pt}
}

% no gap between page line and footnote
%\setlength{\footskip}{17pt}
% https://tex.stackexchange.com/questions/164367/how-to-make-footnotes-appear-at-bottom-of-the-footers-bar

%% START FOOTNOTE FOOTNOTE BELOW HORIZONTAL LINE
\makeatletter
\renewcommand\footnoterule{%
	\kern20\p@
	\hrule\@width\linewidth
	\kern2.6\p@}
\makeatother

\fancypagestyle{phdfancyfootnote}{%
	
	\pagestyle{fancy}
	\fancyhead{} % Clean headers
	\fancyhead[LE,RO]{\THETITLE}
	\fancyhead[RE,LO]{\leftmark}
	\fancyfoot[C]{\thepage}
	\renewcommand{\footrulewidth}{0.0pt}
	\renewcommand{\headrulewidth}{0.4pt}
	\renewcommand{\chaptermark}[1]{\markboth{\thechapter. {\slshape{##1}}}{}} 
}

\fancypagestyle{phdfancyfootnotewithoutheader}{%
	
	\pagestyle{fancy}
	\fancyhead{} % Clean headers
	\fancyhead[LE,RO]{}
	\fancyhead[RE,LO]{}
	\fancyfoot[C]{\thepage}
	\renewcommand{\footrulewidth}{0.0pt}
	\renewcommand{\headrulewidth}{0.0pt}
	\renewcommand{\chaptermark}[1]{\markboth{\thechapter. {\slshape{##1}}}{}} 
}

%% END FOOTNOTE FOOTNOTE BELOW HORIZONTAL LINE

%% BETTER FOOTNOTE WITH FANCY STYLE
\newcommand\footnoteurl[2]{\thispagestyle{empty}\thispagestyle{phdfancyfootnote}\footnote{\href{#2}{#1}}}

\newcommand\footnoteurlsimple[1]{\footnoteurl{#1}{#1}}

\newcommand{\footnoteplain}[1]{\thispagestyle{empty}\thispagestyle{phdfancyfootnote}\footnote{#1}}

%% FOOTNOTE WITHOUT HEADER
\newcommand\footnoteurlwithoutheader[2]{\thispagestyle{empty}\thispagestyle{phdfancyfootnotewithoutheader}\footnote{\href{#2}{#1}}}

\newcommand\footnotewithoutheader[1]{\thispagestyle{empty}\thispagestyle{phdfancyfootnotewithoutheader}\footnote{#1}}

\newenvironment{startpagenumbering}[1][1] {

	%	\pagestyle{plain}
	
	\pagestyle{phdfancy}
	%	\setlength{\footskip}{17pt}
	\pagenumbering{arabic}
	\setcounter{page}{#1}

	% rework plain style to fancy
	\fancypagestyle{plain}{
		
		\pagestyle{fancy}
		\fancypagestyle{phdfancyplain}{%
			\fancyhead{} % Clean headers
			\fancyfoot[C]{\thepage}
			\renewcommand{\footrulewidth}{0.4pt}
			\renewcommand{\headrulewidth}{0.0pt}
		}
		
		\pagestyle{phdfancyplain}
	}

	%% REWORK TITLE CHAPTERS
	\definecolor{gray75}{gray}{0.75}
	\newcommand{\hsp}{\hspace{0pt}}
	\titleformat{\chapter}[hang]{\vspace{-4.8cm}\flushright
		\fontseries{b}\fontsize{80}{100}\selectfont}{\fontseries{b}\fontsize{100}{130}\selectfont \textcolor{gray75}\thechapter\hsp}{0pt}{\\ \vspace{-1cm}\Huge\bfseries}[\titlerule\vspace{-1cm}]

%\titleformat{\chapter}[display]{}{}{2ex}{
%	\begin{center}
%		\hspace*{\dimexpr0em-10pt\relax}
%		\advance\hsize4em\advance\hsize10pt\rule[0.5ex]{4em}{1pt}\hspace{10pt}
%		\begin{varwidth}{\textwidth}\Large\thechapter\end{varwidth}
%		\hspace*{10pt}\rule[0.5ex]{4em}{1pt}
%	\end{center}
%	\vskip 1cm
%	\begin{center}
%		\hspace*{\dimexpr5em-10pt\relax}
%		\Large #1
%	\end{center}
%}

}{\protect\clearpage}


%%% Formulas
\newcommandtwoopt{\KittelEquationExpandedX}[2][Theoretical Kittel equation expanded for a Permalloy thin-film for X-axe][fig:kittel-x]{
        \vspace{-0.5cm}
        \setequation{0cm}{0cm}{
		\small{\[\textit{f} = 28\cdot\sqrt{(B_{DC}+(N_y-N_x)\cdot0.86\cdot10^6\cdot4\pi\cdot10^{-7})\cdot(B_{DC}+(N_z-N_x)\cdot0.86\cdot10^6)\cdot4\pi\cdot10^{-7}}\]}
        }{#2}{#1}
}

\newcommandtwoopt{\KittelEquationExpandedXBoxed}[2][Theoretical Kittel equation expanded for a Permalloy thin-film for X-axe][fig:kittel-x-box]{
	\vspace{-0.5cm}
	\setequationboxed{0cm}{0cm}{
		\colorboxed{red}{11pt}{
			\begin{aligned}
				&f = 28\cdot\sqrt{(B_{DC}+(N_y-N_x)\cdot0.86\cdot10^6\cdot4\pi\cdot10^{-7})\cdot(B_{DC}+(N_z-N_x)\cdot0.86\cdot10^6)\cdot4\pi\cdot10^{-7}} \\ 
				&\text{This line is a comment in boxed formula}
			\end{aligned}
		}
	}{#2}{#1}{0.11cm}
	\vspace{0.8cm}
}

\newcommandtwoopt{\KittelEquationExpandedXBoxedSimple}[2][Theoretical Kittel equation expanded for a Permalloy thin-film for X-axe][fig:kittel-x-box-simple]{
	\vspace{-0.5cm}
	\setequationboxed{0cm}{0cm}{
		\colorboxed{red}{11pt}{
			f = 28\cdot\sqrt{(B_{DC}+(N_y-N_x)\cdot0.86\cdot10^6\cdot4\pi\cdot10^{-7})\cdot(B_{DC}+(N_z-N_x)\cdot0.86\cdot10^6)\cdot4\pi\cdot10^{-7}}
		}
	}{#2}{#1}{0.11cm}
	\vspace{0.8cm}
}

